%%%%%%%%%%%%%%%%%%%%%%%%%%%%%%%%%%%%%%%%%
%  Project4 User Guide 
%
% Important note:
% Chapter heading images should have a 2:1 width:height ratio,
% e.g. 920px width and 460px height.
%
%%%%%%%%%%%%%%%%%%%%%%%%%%%%%%%%%%%%%%%%%

%----------------------------------------------------------------------------------------
%	PACKAGES AND OTHER DOCUMENT CONFIGURATIONS
%----------------------------------------------------------------------------------------

\documentclass[11pt,fleqn]{book} % Default font size and left-justified equations

\usepackage[top=3cm,bottom=3cm,left=3.2cm,right=3.2cm,headsep=10pt,letterpaper]{geometry} % Page margins

\usepackage{xcolor} % Required for specifying colors by name
\definecolor{ocre}{RGB}{52,177,201} % Define the orange color used for highlighting throughout the book

\usepackage{booktabs}

% Font Settings
\usepackage{avant} % Use the Avantgarde font for headings
%\usepackage{times} % Use the Times font for headings
\usepackage{mathptmx} % Use the Adobe Times Roman as the default text font together with math symbols from the Sym­bol, Chancery and Com­puter Modern fonts

\usepackage{microtype} % Slightly tweak font spacing for aesthetics
\usepackage[utf8]{inputenc} % Required for including letters with accents
\usepackage[T1]{fontenc} % Use 8-bit encoding that has 256 glyphs

\usepackage{menukeys}

% Bibliography
\usepackage[style=alphabetic,sorting=nyt,sortcites=true,autopunct=true,babel=hyphen,hyperref=true,abbreviate=false,backref=true,backend=biber]{biblatex}
\addbibresource{bibliography.bib} % BibTeX bibliography file
\defbibheading{bibempty}{}

\input{structure} % Insert the commands.tex file which contains the majority of the structure behind the template

\begin{document}

%----------------------------------------------------------------------------------------
%	TITLE PAGE
%----------------------------------------------------------------------------------------

\begingroup
\thispagestyle{empty}
\center{\includegraphics[scale=0.25]{badgecolour}} % Thirsty Goat Badge 
%\AddToShipoutPicture*{\put(0,0){\includegraphics[scale=1.25]{esahubble}}} % Image background
\centering
\vspace*{5cm}
\par\normalfont\fontsize{35}{35}\sffamily\selectfont
\textbf{Project4}\\
{\LARGE Project4 tagline here}\par % Book title
\vspace*{1cm}
{\Huge User Guide}\par % Author name
\endgroup

%----------------------------------------------------------------------------------------
%	TABLE OF CONTENTS
%----------------------------------------------------------------------------------------

\thispagestyle{empty}
\tableofcontents % Print the table of contents itself

%----------------------------------------------------------------------------------------
%	CHAPTER 1
%----------------------------------------------------------------------------------------

\chapter{Introduction}
\thispagestyle{empty}

\section{Parts of the Main Window}
The main Project4 workspace is show in figure~\ref{fig:mainwindow}.

\begin{figure}[h]
  \centering
  \includegraphics[scale=0.32]{mainwindow}
  \caption{The main Project4 window.\label{fig:mainwindow}}
\end{figure}

By default, the main window is split into two panes.
On the left is the list pane.
On the right is a simple text field.
To display or hide the list pane, choose \menu{View > List}.
Alternatively, use the keys \keys{\ctrl + L}.\footnote{On Mac,
use the command key \keys{\cmd} in place of \keys{\ctrl}.}

\section{Creating a New Project}
\begin{itemize}
  \item From the menu bar, use \menu{File > New > Project}.
  \item Use the keyboard shortcut \keys{\ctrl + N}.
\end{itemize}
You will then be prompted to enter the details of the new project.

\begin{figure}[h]
  \centering
  \includegraphics[scale=0.32]{newproject}
  \caption{The new project window.\label{newproject}}
\end{figure}

\section{Opening an Existing Project}
\textbf{Note:} Only one project can be open at a time.
\begin{itemize}
  \item From the menu bar, use \menu{File > Open}.
  \item Use the keyboard shortcut \keys{\ctrl + O}.
\end{itemize}

\section{Saving a Project}
\begin{itemize}
  \item From the menu bar, use \menu{File > Save}.
  \item Use the keyboard shortcut \keys{\ctrl + S}.
\end{itemize}

\section{Closing Project4}
  \begin{itemize}
    \item \menu{File > Quit}.
  \end{itemize}

\appendix
\chapter{Appendix}

\section{Keyboard Shortcuts}

\textbf{Note:} On Mac the command key \keys{\cmd} should be used
  in place of \keys{\ctrl}.

\begin{table}[h]
  \renewcommand{\arraystretch}{1.5} % Increase space between rows
  \centering
  \begin{tabular}{lc}
    \toprule
    Action & Shortcut \\
    \midrule
    Hide/Show List Pane  &  \keys{\ctrl + L} \\ 
    Create New Project & \keys{\ctrl + N} \\
    Open Project & \keys{\ctrl + O} \\ 
    Save Project & \keys{\ctrl + S} \\
    \bottomrule
  \end{tabular}
  \caption{List of Keyboard Shortcuts}
  \label{tab:shortcuts}
\end{table}

\end{document}
